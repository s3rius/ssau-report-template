%------------------------------------------------
% Quick reference. 
%------------------------------------------------
%
% Для вставки картинок:
%
%--------         Комманда
%
%\begin{figure}[H]
%	\includegraphics{img_name}
%	\caption{some caption}
%	\label{some_pic}
%\end{figure}
%
%--------        Переменные
%
% img_name     <- Название картинки в папке img.
% some_caption <- подпись картинки.
% label        <- лейбл нужен для ссылок на картинку.
% H            <- расположение картинки на странице.
%
%--------         Пример
%
%\begin{figure}[H]
%	\includegraphics{pic1.jpg}
%	\caption{График}
%	\label{grapics1}
%\end{figure}
%
%------------------------------------------------
%
% Для референса по лейблу:
%
%--------         Комманда
%
% Для ссылки используется \eqref{ref}.
%
%--------        Переменные
%
% ref          <- указанный лейбл в директиве \label{ref}
%                 Ссылку можно сделать на любой объект имеющий \label{•}
%
%--------         Пример
%
% \eqref{graphics1}
%
%------------------------------------------------
%
% Для листинга кода:
%
%--------         Комманда
%
% \lstinputlisting[language=lang,mathescape=true]{src}
%--------        Переменные
%
% lang         <- язык на котором написан исходный код, например "python" или "C++".
% mathescape   <- если в исходниках есть формулы LaTeX, то они будут представлены как формулы.
% src          <- путь до файла исходников.
%
%--------         Пример
%
% \lstinputlisting[language=C++,mathescape=false]{./src/main.cpp}
%
%------------------------------------------------
%
% Для вставки таблиц:
%
%--------
%\begin{table}[H]
%	\centering
%	\caption{ capt }
%	\begin{tabularx}{0.9\textwidth}{ | Y | Y | }
%		\hline
%		lines
%	\end{tabularx}
%	\label{tab1}
%\end{table}
%--------
% caption      <- Подпись таблицы.
% tab1         <- лейбл нужный для ссылки на таблицу.
% | Y | Y |    <- количество и формат столбцов.
% Y            <- Тип столбца.
%                 В данном случае определены кастомные столбцы Y (Спасибо Максиму Наумову)
% |            <- обозначает границы столбца.
%                 То есть, если будет указано |Y Y|, то столбцы внутри строк разделены не будут.
% H            <- То же самое, что и у картинок.
% lines        <- непосредственно элементы таблицы.
%                 Разделяются знаком "&", оканчивать каждую строку лучше \\ \hline
%
%--------         Пример
%\begin{table}[H]
%	\centering
%	\caption{ capt }
%	\begin{tabularx}{0.9\textwidth}{ | Y | Y | }
%		\hline
%		str1 & str2 \\ \hline
%		str1 & str2 \\ \hline
%		str1 & str2 \\ \hline
%		str1 & str2 \\ \hline
%		str1 & str2 \\ \hline
%	\end{tabularx}
%	\label{tab1}
%\end{table}
%------------------------------------------------

\documentclass[12pt, fleqn, a4paper]{extarticle}

\makeatletter
\renewcommand*\l@section{\@dottedtocline{1}{1.5em}{2.3em}}



%\includegraphics{universe}

\usepackage[utf8]{inputenc}
\usepackage[T2A]{fontenc}
\usepackage[russian]{babel} % указывает язык документа
\usepackage[left=3cm,right=2cm,top=2cm,bottom=2cm,bindingoffset=0cm]{geometry}
\usepackage{lastpage}
\usepackage{fancyhdr}
\usepackage{titlesec}
\usepackage{graphicx} % для вставки картинок
\usepackage[intlimits]{mathtools} % математические дополнения
\usepackage{amssymb}
\usepackage[tableposition=top]{caption}
\usepackage{subcaption}
\usepackage{indentfirst}
\usepackage{minted}
\usepackage{tabularx}
\usepackage{tabulary}
\usepackage{multirow}
\usepackage{float}
\usepackage[figure,table]{totalcount}
\usepackage{diagbox}
\usepackage[german=guillemets]{csquotes}
\usepackage{fontspec} 
\usepackage{enumitem}
%\usepackage{mathptmx}% http://ctan.org/pkg/mathptmx
%\usepackage{showframe}
\usepackage{hyperref}

\setlength{\parindent}{1.2cm}

\setlength{\mathindent}{1.2cm}

\defaultfontfeatures{Ligatures={TeX},Renderer=Basic} 
\setmainfont[Ligatures={TeX,Historic}]{Times New Roman}

%\setlist[enumerate]{itemindent=\dimexpr\labelwidth+\labelsep\relax,leftmargin=0pt}

%\setlength{\section*}{0.5cm}
%\usepackage{minted}
%\usepackage{fancyvrb}
%\usepackage{newtxtext}

%\titleformat{\section}[hang]{\bfseries\LARGE\centering}{}{1em}{}

%\setlist[enumerate]{itemindent=\dimexpr\labelwidth+\labelsep\relax,leftmargin=0pt}
\setlist[enumerate,itemize]{leftmargin=0pt,itemindent=1.7cm}

\titleformat{\section}{\large\bfseries\centering}{\thesection}{0.5em}{\MakeUppercase}
\titleformat{\subsection}[block]{\bfseries\hspace{1em}}{\thesubsection}{0.5em}{}
%\setlength{\subsection*}{1.5cm}
%\setlength{\parindent}{4em}

%\setlength{\parindent}{1.5cm}

\captionsetup[figure]{labelfont={it},textfont={it},name={Рисунок},labelsep=endash, skip=5pt}
\captionsetup[table]{labelfont={it},textfont={it},name={Таблица},labelsep=endash,singlelinecheck=false, skip=5pt, margin=1cm}


%\renewcommand{\baselinestretch}{1.5}
\linespread{1.5} % полуторный интервал
\frenchspacing
\graphicspath{ {src/images/} }

  %-------------------------------------------
  % Переменные
  %-------------------------------------------

  \newcommand{\firstAuthorSurName}{Фамилия} 					  % Фамилия автора.
  \newcommand{\firstAuthorInitials}{ А. Б.} 					  % Фамилия автора.
  \newcommand{\teacherName}{Преподаватель А. Б.}				  % Имя преподавателя.
  \newcommand{\variantNumber}{var} 							      % Номер варианта.
  \newcommand{\labNumber}{num} 							          % Номер варианта.
  \newcommand{\groupNumber}{номер-группы} 				          % Номер группы.
  \newcommand{\subjectTitle}{Название предмета}                   % Название предмета.
  \newcommand{\taskTitle}{Название лабораторной работы} 		  % Название работы.

  
  %-------------------------------------------
  % Ссылки в оглавлении
  %-------------------------------------------
  

\hypersetup{
    colorlinks,
    citecolor=black,
    filecolor=black,
    linkcolor=black,
    urlcolor=black
}

  %-------------------------------------------
  % Стиль футеров и хедеров
  %-------------------------------------------

\pagestyle{fancy}
\fancyhead[L, R]{}
\fancyfoot[L]{}
\fancyfoot[R]{}
\renewcommand{\footrulewidth}{0pt}
\renewcommand{\headrulewidth}{0pt}

%\renewcommand\subsectionfont{\normalfont\normalsize\bfseries}

\def\l@subsection{\@dottedtocline{2}{3.8em}{3.2em}}

\newcolumntype{Y}{>{\centering\arraybackslash}X}

\begin{document}

%----------------------------------------------------------------------------------------
%	TITLE PAGE
%----------------------------------------------------------------------------------------
\pagenumbering{Alph}

\begin{titlepage}
							
	\center
							
	%------------------------------------------------
	%	Заголовки
	%------------------------------------------------
							
	\textsc{Министерство образования и науки Российской Федерации}\\[-0.15cm]
	\textsc{Федеральное государственное автономное образовательное учреждение \\[-0.15cm] высшего образования}\\[-0.15cm] 
	\textsc{«Самарский национальный исследовательский университет \\[-0.15cm] имени академика С.П.Королёва»}\\[0.5cm]
	\textsc{Институт информатики, математики и электроники}\\[-0.7em]
	\textsc{Факультет информатики}\\[-0.7em]
	\textsc{Кафедра технической кибернетики}\\[-1em]
						
	%------------------------------------------------
	%	Название работы
	%------------------------------------------------
							
	\vfill\vfill
						    
							
	\textbf{Отчет по лабораторной работе №\labNumber}\\[-0.7em]
	
	\textbf{по курсу «\subjectTitle»}\\
	
	\large Тема: \enquote{\textbf{\taskTitle}}\\[0.5cm]
	
    \vfill
    
    Вариант № \variantNumber\\[0.5cm]
    
    \vfill\vfill\vfill\vfill
							
	\begin{minipage}{1\textwidth}
		\begin{center}
			\begin{tabularx}{\textwidth}{X l}
				Выполнил студент:        & \firstAuthorSurName \firstAuthorInitials \\
				Группа:                  & \groupNumber                     		           \\
				Преподаватель:           & \teacherName         		                \\
			\end{tabularx}
		\end{center}
	\end{minipage}
							
						
	%------------------------------------------------
	%	Дата
	%------------------------------------------------
							
	\vfill\vfill\vfill
					
	{\centering Самара \the\year}
							
							
\end{titlepage}

\pagenumbering{arabic}

\setcounter{page}{2}


%------------------------------------------------
%Задание
%------------------------------------------------

\section*{Задание}
\begin{enumerate}
 \item Задача 1;
 \item Задача 2;
 \item Задача 3.
\end{enumerate}

\newpage

%------------------------------------------------
% Начало основной части
%------------------------------------------------
\titleformat{\section}{\large\bfseries}{\thesection}{0.5em}{}
\titlespacing*{\section}{\parindent}{1ex}{1em}
\section{Основная часть}

Описание проделанной работы.

\begin{figure}[H]
	\includegraphics[height=0.6\textwidth]{plot.png}
	\caption{Описание изображения}
	\label{some_pic}
\end{figure}


\newpage

%------------------------------------------------
% Приложения. Коды программ и.т.д.
%------------------------------------------------
\titleformat{\section}{\large\bfseries\centering}{\thesection}{0.5em}{\MakeUppercase}
\titleformat{\subsection}[block]{\bfseries\hspace{1em}}{\thesubsection}{0.5em}{}

\section*{Приложение А}
\addcontentsline{toc}{section}{Приложение А Код программы}
\begin{center}
\textbf{Код программы}
\end{center}
\inputminted[breaklines, mathescape, fontsize=\footnotesize]{rust}{./src/script.py}


\end{document}
